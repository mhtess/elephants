% 
% Annual Cognitive Science Conference
% Sample LaTeX Paper -- Proceedings Format
% 

% Original : Ashwin Ram (ashwin@cc.gatech.edu)       04/01/1994
% Modified : Johanna Moore (jmoore@cs.pitt.edu)      03/17/1995
% Modified : David Noelle (noelle@ucsd.edu)          03/15/1996
% Modified : Pat Langley (langley@cs.stanford.edu)   01/26/1997
% Latex2e corrections by Ramin Charles Nakisa        01/28/1997 
% Modified : Tina Eliassi-Rad (eliassi@cs.wisc.edu)  01/31/1998
% Modified : Trisha Yannuzzi (trisha@ircs.upenn.edu) 12/28/1999 (in process)
% Modified : Mary Ellen Foster (M.E.Foster@ed.ac.uk) 12/11/2000
% Modified : Ken Forbus                              01/23/2004
% Modified : Eli M. Silk (esilk@pitt.edu)            05/24/2005
% Modified : Niels Taatgen (taatgen@cmu.edu)         10/24/2006
% Modified : David Noelle (dnoelle@ucmerced.edu)     11/19/2014
% Modified : Roger Levy (rplevy@mit.edu)     12/31/2018



%% Change "letterpaper" in the following line to "a4paper" if you must.

\documentclass[10pt,letterpaper]{article}

\usepackage{cogsci}

%\cogscifinalcopy % Uncomment this line for the final submission 


\usepackage{pslatex}
\usepackage{apacite}
\usepackage{float} % Roger Levy added this and changed figure/table
                   % placement to [H] for conformity to Word template,
                   % though floating tables and figures to top is
                   % still generally recommended!

%\usepackage[none]{hyphenat} % Sometimes it can be useful to turn off
%hyphenation for purposes such as spell checking of the resulting
%PDF.  Uncomment this block to turn off hyphenation.


%\setlength\titlebox{4.5cm}
% You can expand the titlebox if you need extra space
% to show all the authors. Please do not make the titlebox
% smaller than 4.5cm (the original size).
%%If you do, we reserve the right to require you to change it back in
%%the camera-ready version, which could interfere with the timely
%%appearance of your paper in the Proceedings.

\providecommand{\tightlist}{%
  \setlength{\itemsep}{0pt}\setlength{\parskip}{0pt}}
  
\newcommand{\denote}[1]{\mbox{ $[\![ #1 ]\!]$}}

\title{Elephants live in Africa and Asia}
 
\author{{\large \bf Michael Henry Tessler (tessler@mit.edu)} \\
  Department of Brain and Cognitive Sciences, MIT \\
  Cambridge, MA 02139 USA
  \AND {\large \bf Roger Levy (SDJ@Macc.Wisc.Edu)} \\
  Department of Brain and Cognitive Sciences, MIT \\
  Cambridge, MA 02139 USA}


\begin{document}

\maketitle


\begin{abstract}

\textbf{Keywords:} 
semantics; pragmatics; incremental processing; generics
\end{abstract}


\section{Introduction}

Extending an analysis of generics to handle complex-predicates poses unique challenges.
A sentence of the form ``Ks F and G'' introduces an ambiguity:

\begin{enumerate}
\item $\denote{gen}(K) [F \land G]$
\item $\denote{gen}(K) [F] \land \denote{gen}(K) [G]$
\end{enumerate}

Consider the following examples:

\begin{enumerate}
\tightlist
\item Triangles have three sides and three angles.
\item Ravens have two wings and two legs.
\item Elephants live in Africa and Asia.
\item Lions have manes and give live birth.
\end{enumerate}

It is conceivable that both (1) and (2) can be analyzed in terms of a (context-sensitive) generic operator acting on a logically complex predicates (i.e., \emph{all triangles both have three sides and three angles}, \emph{most ravens have two wings and two legs}). 
Individual elephants do not migrate across continents and thus live in both Africa and Asia. 
The individual lions that have manes are a distinct sub-category from those that give live birth (i.e., the properties pick out the males and the females, respectively). 
The logically complex predicates in sentences (3) and (4), thus, are not true generically of their associated categories; rather, the sentences should be understood as a conjunction of generic sentences.

\section{Computational Model}


\section{Experiments}


\section{Acknowledgments}



\bibliographystyle{apacite}

\setlength{\bibleftmargin}{.125in}
\setlength{\bibindent}{-\bibleftmargin}

\bibliography{elephants}


\end{document}
