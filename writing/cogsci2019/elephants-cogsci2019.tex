% 
% Annual Cognitive Science Conference
% Sample LaTeX Paper -- Proceedings Format
% 

% Original : Ashwin Ram (ashwin@cc.gatech.edu)       04/01/1994
% Modified : Johanna Moore (jmoore@cs.pitt.edu)      03/17/1995
% Modified : David Noelle (noelle@ucsd.edu)          03/15/1996
% Modified : Pat Langley (langley@cs.stanford.edu)   01/26/1997
% Latex2e corrections by Ramin Charles Nakisa        01/28/1997 
% Modified : Tina Eliassi-Rad (eliassi@cs.wisc.edu)  01/31/1998
% Modified : Trisha Yannuzzi (trisha@ircs.upenn.edu) 12/28/1999 (in process)
% Modified : Mary Ellen Foster (M.E.Foster@ed.ac.uk) 12/11/2000
% Modified : Ken Forbus                              01/23/2004
% Modified : Eli M. Silk (esilk@pitt.edu)            05/24/2005
% Modified : Niels Taatgen (taatgen@cmu.edu)         10/24/2006
% Modified : David Noelle (dnoelle@ucmerced.edu)     11/19/2014
% Modified : Roger Levy (rplevy@mit.edu)     12/31/2018



%% Change "letterpaper" in the following line to "a4paper" if you must.

\documentclass[10pt,letterpaper]{article}

\usepackage{cogsci}
\usepackage{color}

%\cogscifinalcopy % Uncomment this line for the final submission 


\usepackage{pslatex}
\usepackage{apacite}
\usepackage{float} % Roger Levy added this and changed figure/table
                   % placement to [H] for conformity to Word template,
                   % though floating tables and figures to top is
                   % still generally recommended!

%\usepackage[none]{hyphenat} % Sometimes it can be useful to turn off
%hyphenation for purposes such as spell checking of the resulting
%PDF.  Uncomment this block to turn off hyphenation.


%\setlength\titlebox{4.5cm}
% You can expand the titlebox if you need extra space
% to show all the authors. Please do not make the titlebox
% smaller than 4.5cm (the original size).
%%If you do, we reserve the right to require you to change it back in
%%the camera-ready version, which could interfere with the timely
%%appearance of your paper in the Proceedings.

\providecommand{\tightlist}{%
  \setlength{\itemsep}{0pt}\setlength{\parskip}{0pt}}
  
\definecolor{Red}{RGB}{255,0,0}
\definecolor{Green}{RGB}{10,200,100}
\definecolor{Blue}{RGB}{10,100,200}

\newcommand{\mh}[1]{{\textcolor{Blue}{[mh: #1]}}}
\newcommand{\rl}[1]{{\textcolor{Green}{[roger: #1]}}}

\newcommand{\denote}[1]{\mbox{ $[\![ #1 ]\!]$}}
\newcommand{\red}[1]{{\textcolor{Red}{#1}}}

\title{Incremental, non-monotonic understanding of conjunctive generic sentences}
 
\author{{\large \bf Michael Henry Tessler (tessler@mit.edu)} \\
  Department of Brain and Cognitive Sciences, MIT \\
  Cambridge, MA 02139 USA
  \AND {\large \bf Roger Levy (SDJ@Macc.Wisc.Edu)} \\
  Department of Brain and Cognitive Sciences, MIT \\
  Cambridge, MA 02139 USA}


\begin{document}

\maketitle


\begin{abstract}

\textbf{Keywords:} 
semantics; pragmatics; incremental processing; generics
\end{abstract}


\section{Introduction}

Much of what we come to learn about the world comes not from direct experience but from knowledge conveyed to us from others, often in the form of linguistic utterances. 
``Elephants eat 300 pounds of a food in a day'' succinctly conveys information that extends beyond any specific context (e.g., it could apply to any elephant, any day of the week). 
Utterances that communicate generalizations are called \emph{generic} utterances \cite{Carlson1977, genericBook}, and they are the foremost case study of rich, abstract knowledge conveyed in simple utterances \cite{Gelman2009}.
%Understanding how generic language updates beliefs holds the key to understanding how young children build rich, intuitive theories \cite{Gelman2009}. 

Understanding generic language pose a number of philosophical challenges that have made it very difficult to develop a unified, formal theory of their meaning \cite<for useful reviews, see:~>{genericBook, Leslie2008, Nickel2016}. 
The sentence ``Robins lay eggs'' is true despite the property only applying to female robins, yet the same logic does not make ``Robins are female'' a true or felicitous utterance. 
``Elephants live in Africa and Asia'' is true despite no elephant living in both; the sentence should be understood as ``Elephants live in Africa, and elephants live in Asia'', but this is impossible if each individual generic sentence means something analogous to \emph{most} or \emph{all} \cite{Nickel2008}.


%Crucially, this theory posits that the meaning of a generic is uncertain one which can be updated as more information is available.

The puzzles of generic language deepen when one considers that actual linguistic input needs to be processed incrementally: Sentences take time to say and written language is generally read from beginning to end. 
Listeners comprehend sentences incrementally, making use of moment to moment linguistic input in order to make predictions about future input \cite<e.g.,>{Altmann1999}.
The sentence  ``Elephants live in Africa and Asia'', thus, at some point is heard as just ``Elephants live in Africa'', which could imply a different meaning than the sentence as a whole.


Here, we combine a recent model of generic language understanding  \cite{Tessler2019:genLang} with an incremental parsing mechanism to begin to form beliefs about a speaker's intended meaning before the speaker has finished speaking.
The basic, non-incremental model understands ``Elephants live in Africa and Asia” as meaning that some elephants live in Africa and that different ones live in Asia, in the case where listeners have prior knowledge suggesting against the existence of international elephants.
Furthermore, the enriched incremental model makes the prediction that part-way through generic sentences about conjunctive predicates, a listener will form strong beliefs which later get non-monotonically updated once more information comes in.
We test this prediction in a set of language understanding tasks where participants are asked to report their beliefs about the prevalence of a property in a category at different points in a sentence, analogous to gating paradigms in psycholinguistics \cite{Grosjean1980}.


%Recently, \cite{Tessler2019:genLang} proposed a theory of generic language wherein the meaning of a generic is \emph{underspecified} (or, vague) and Bayesian reasoning is used to resolve more precise meanings in context.


%Intriguingly, this model also predicts that (i.e., upon hearing only that ``Elephants live in Africa'') leads our model to believe that most, possibly all, elephants live in Africa, but when the sentence is completed (``. . . and Asia''), the model nonmonotonically updates its beliefs to the weaker \emph{some elephants live in Africa and others in Asia}. 


%Interpreting generics in a consistent way is a non-starter. 


%Theories that appeal to standard, formal semantic, quantificational truth conditions (e.g., generic means \emph{most} or \emph{all} relevant or normal members of the category have the property) employ mechanisms to implicitly restrict the \emph{relevant} set of robins to be females in the case of a property concerning reproduction such as \emph{laying eggs}\cite<e.g.,>{Cohen1999}. 
%(such as that of  where a generic ``Ks F'' means roughly that \emph{most of the relevant Ks F})

%Interpreting generics in a 
%Such restrictions, however, are too limiting when interpreting conjunctive predicates as in ``Elephants live in Africa and Asia''. 
% from knowledge about the kind of property under consideration (i.e., \emph{laying eggs} has to do with reproduction, so we are only talking about females), but this mechanism is too restrictive when it comes to generics about conjunctive properties such as ``Elephants live in Africa and Asia'': 
%It cannot be the case that most (more than half of) elephants live in Africa and most (more than half of) elephants live in Asia, unless we imagine elephants migrating back-and-forth from continent to continent (i.e,. \emph{international elephants}), which is intuitively implausible \cite{Nickel2008}.
%More generally, a theory of generics need be sufficiently flexible to accommodate property-specific interpretations (e.g., ``Robins lay eggs'' means half of robins lay eggs; ``Dogs get cancer'' means some dogs get cancer; ``Birds fly'' means almost all birds fly) as well as be able to revise those interpretations with new, potentially conflicting evidence (e.g., consider ``Elephants live in Africa''~vs.~``Elephants live in Africa and Asia'').



%``Lions have manes and give live birth'' wherein the first and second conjuncts apply to distinct subsets (i.e., male lions have manes and female lions give live birth).



%Consider the following examples:
%
%\begin{enumerate}
%\tightlist
%\item Triangles have three sides and three angles.
%\item Ravens have two wings and two legs.
%\item Elephants live in Africa and Asia.
%\item Lions have manes and give live birth.
%\end{enumerate}
%
%It is conceivable that both (1) and (2) can be analyzed in terms of a (context-sensitive) generic operator acting on a logically complex predicates (i.e., \emph{all triangles both have three sides and three angles}, \emph{most ravens have two wings and two legs}). 
%Individual elephants do not migrate across continents and thus live in both Africa and Asia. 
%The individual lions that have manes are a distinct sub-category from those that give live birth (i.e., the properties pick out the males and the females, respectively). 
%The logically complex predicates in sentences (3) and (4), thus, are not true generically of their associated categories; rather, the sentences should be understood as a conjunction of generic sentences.

\section{Computational Model}

Extending an analysis of generics to handle complex-predicates poses unique challenges.
A sentence of the form ``Ks F and G'' introduces an ambiguity:

\begin{enumerate}
\tightlist
\item $\denote{gen}(K) [F \land G]$
\item $\denote{gen}(K) [F] \land \denote{gen}(K) [G]$
\end{enumerate}


\section{Experiment}

\subsection{Methods}

\subsection{Participants}
We recruited N participants through Amazon's Mechanical Turk.
Participants were restricted to those with verified U.S. IP addresses and had at least a 95\% work approval rating. 

\subsection{Procedure}
Participants were told they would be reading a storybook and be asked questions at the end of each chapter. 
Questions were all of the same type, an \emph{implied prevalence} question \cite{Gelman2002, Cimpian2010} that read: ``Out of all Ks, what percentage do you think F?'', where K represents a category and F a feature. 
Responses were recorded using a 101-pt slider bar with end-points labeled 0\% and 100\%, and with the exact value appearing dynamically above the slider. 
Participants were familiarized with the response variable in a practice trial, where they were asked to report how many dogs bark, birds are male, and ticks carry lyme disease. 
We used these questions to encourage participants to use the full range of the response scale. 

Following this instructions / familiarization period, participants read a storybook consisting of 16 paragraphs we refer to as \emph{chapters} (items), each of which spanned 2 - 4 screens of text which spanned 2 - 4 lines on the screen. 
At the end of all but the first chapter, participants answered an implied prevalence question relating to a sentence in the preceding chapter. 
At the end of the storybook, participants completed a memory check question where they had to select all the facts they had learned in the story from a list of 10 (5 real, 5 distractor); in addition, participants were asked to explain what the experiment was about in very general terms.

We used 10 critical chapters (items), each which could be assigned to one of two experimental conditions.
A given chapter in the two experimental conditions differed only in the information present on the final screen of a chapter.
The penultimate screen of all critical trials ended with the beginning of a conjunctive sentence that was broken up immediately before the ``and'' in the sentence \red{(Figure); e.g., ``They ascribe to the Caboo religion'')}. 
Text on the page was fully-justified so that the final word on a page always appeared in the bottom-right corner of the screen, giving the appearance that the page naturally ended at that word.
In the \emph{relevant conjunction} condition, the continuation of the sentence (final screen) was a property that was anti-correlated (or, somewhat mutually exclusive) with the property on the preceding page (e.g., ``and the Daith religion''). 
In the \emph{irrelevant conjunction} condition, the continuation was a property that was unrelated to the property on the preceding page (e.g., ``...and follow a strict code of laws.''). 
In addition, we had 5 filler trials which appeared similar to the \emph{irrelevant conjunction} cases.

\subsection{Materials}

The storybook was concerned with aliens and animals on a far-away planet and read like a children's book without any narrative (each chapter concerned non-interactiving categories and characters).
The text of a chapter mostly composed of generic sentences about novel categories (e.g., ``Glippets are intelligent creatures''), though some sentences mentioned specific fictional characters (``Wint lived a long time ago in the mountains.'').

Each chapter of the storybook was on




\section{Acknowledgments}



\bibliographystyle{apacite}

\setlength{\bibleftmargin}{.125in}
\setlength{\bibindent}{-\bibleftmargin}

\bibliography{elephants}


\end{document}
